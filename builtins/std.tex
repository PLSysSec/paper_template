\usepackage{amsmath,amssymb,amsfonts,amsthm} % Lots of useful math, symbols, etc.
\usepackage{comment}                         % Comment environment---nothing within is typeset
\usepackage{ifthen}                          % One of many approaches to simple conditionals when writing macros
\usepackage{minibox}                         % Alignment/frame-supporting wrapper around \vbox{\hbox{...}...\hbox{...}}
\usepackage{multirow}                        % Row-spanning table cells
\usepackage{suffix}                          % Macros with non-alphabetic suffixes (Requires e-TeX)
\usepackage{graphicx}                        % \includegraphics command to include images
\usepackage[hyphens]{url}                    % \url command
\usepackage{xargs}                           % \newcommandx command for defining commands with default args
\usepackage[dvipsnames]{xcolor}              % colors!
\usepackage[colorlinks=true,linkcolor=Green,citecolor=NavyBlue]{hyperref} % Hyperlinks
\usepackage[capitalise]{cleveref}            % \cref for automatically using the right word for refs
\usepackage{hyphenat}                        % \hyp{} for line-wrappable hyphen, plus redefines \_ to be line-wrappable
\usepackage{subcaption}                      % \subcaption command for captioning subfigures
\usepackage{fancyvrb}                        % Verbatim environment: less brittle than \verb
\usepackage{supertabular}                    % supertabular environment: allows tables to flow to next page
\usepackage{xspace}                          % fixes space after text-only macros
\usepackage{microtype}                       % typographical refinements

% Hyperlink wrapping
\hypersetup{breaklinks=true}

% Proper english
\newcommand\ie{i.e.,\xspace}
\newcommand\eg{e.g.,\xspace}
\newcommand\Ie{I.e.,\xspace}
\newcommand\Eg{E.g.,\xspace}

% Drafting macros
\newcommand{\todo}[1]{{\color{red}#1}}
\newcommand{\redo}[1]{{\color{violet}#1}}
% adding a prefix else its too easy to miss
\newcommand{\tocite}[1]{{\color{red}[TOCITE:#1]}}
\newcommand{\ignore}[1]{}

% Paragraph "subsubsubsection"
\newcommand{\para}[1]{\smallskip\noindent\textbf{{#1.}\xspace}}

% A fancy dash
\def\dash---{\kern.16667em---\penalty\exhyphenpenalty\hskip.16667em\relax}

% A compact list
\newenvironment{CompactItemize}%
  {\begin{list}{$\blacktriangleright$}%
    {\leftmargin=\parindent \itemsep=2pt \topsep=2pt
     \parsep=0pt \partopsep=0pt}}%
  {\end{list}}
\renewcommand{\labelitemi}{$\blacktriangleright$}
